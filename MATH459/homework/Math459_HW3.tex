\documentclass[12pt]{article}

\setlength{\evensidemargin}{-0.25 in}
\setlength{\oddsidemargin}{-0.25 in} \setlength{\textwidth}{6.8 in}
\setlength{\topmargin}{-0.5 in} \setlength{\textheight}{8.6 in}
\setlength{\parindent}{0in}
\setlength{\parskip}{\baselineskip}
\usepackage{hyperref}
\usepackage{amsmath}
\usepackage{amssymb}
\usepackage{amsthm} 
\usepackage{xy}
\usepackage{verbatim}
\usepackage[english]{babel}
\usepackage{graphicx}
\newtheorem{definition}{Definition}
\usepackage{bbm}


\begin{document}
\vspace{-2cm}
\begin{center}
{\large \bf Math 459 HW3} \\
{\it Due Thursday, March 31} \\
\end{center}

\noindent
\textbf{\underline{Guidelines:}}
\begin{itemize}
\item You must show your work to get credit.
\item Include your \textbf{R} code and the output (just copy+paste into a text file).
\end{itemize}


\textbf{1.} In this homework you will perform a Bayesian analysis of the gamma distribution using an uninformative prior and MCMC. The density of the gamma distribution is
$$
f(x | \alpha, \beta)=\frac{\beta^{\alpha}}{\Gamma(\alpha)}x^{\alpha-1}\exp(-\beta x), \;\; \alpha, \beta > 0.
$$
Derive Jeffreys prior (you must show your work to get credit). Some hints:
\begin{itemize}
\item To find the Fisher information matrix, you only need the log-likelihood for a sample of size 1. 
\item The derivatives of the natural logarithm of the gamma function are special functions. Note that $\partial\log(\Gamma(\alpha))/\partial \alpha$ is the \texttt{digamma} function (also in \texttt{R}) and $\partial^{2}\log(\Gamma(\alpha))/\partial \alpha^{2}$ is the \texttt{trigamma} function.
\end{itemize}



\bigskip
\textbf{2.} Write your own function to perform random walk Metropolis-Hastings sampling (with $10,000$ samples) from a density which is proportional to the above gamma density times Jeffreys prior. To get full credit, you must add comments to each step of the code to explain what is happening. Some guidelines:
\begin{itemize}
\item Remember that the default parameters for the \texttt{gamma} distribution in \texttt{R} are not the same as the usual gamma density above. Make sure you specify the gamma density as above in \texttt{R}.
\item You can use any symmetric proposal density that you want.
\item Make sure to save the output.
\end{itemize}


\bigskip
\textbf{3.} Use the \texttt{coda package} to give traceplots, autocorrelation function plots and perform all 4 diagnostic checks in the lecture notes (Gelman \& Rubin, Geweke, Raftery \& Lewis, and Heidelberg \& Welch). \underline{Interpret} these results.


\bigskip
\textbf{4.} Plot the MCMC estimate of the marginal posterior density for each parameter (using the sampled values). Give an (estimated) 95\% HPD interval for each parameter.
\end{document}